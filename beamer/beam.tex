\documentclass[ignorenonframetext,]{beamer}
\usepackage{amssymb,amsmath}
\usepackage{ifxetex,ifluatex}
\usepackage{fixltx2e} % provides \textsubscript
\usepackage{array}
\usepackage{graphicx}
\usepackage{adjustbox}
\ifxetex
  \usepackage{fontspec,xltxtra,xunicode}
  \defaultfontfeatures{Mapping=tex-text,Scale=MatchLowercase}
\else
  \ifluatex
    \usepackage{fontspec}
    \defaultfontfeatures{Mapping=tex-text,Scale=MatchLowercase}
  \else
    \usepackage[utf8]{inputenc}
  \fi
\fi

%\usetheme{Boadilla} %pointless
%\usetheme{CambridgeUS} %pointless
%\usetheme{Montpellier} %pointless
%\usetheme{Dresden} %good but vertically short
\usetheme{Darmstadt} %GOOD
%\usetheme{Berlin}
%\usetheme{Warsaw}
%\usetheme{Frankfurt}
%\usecolortheme{seahorse}
%\usecolortheme{fly}  % GREY
%\usecolortheme{seagull}
%\usecolortheme{rose}
%\usecolortheme{orchid}
\usefonttheme{structurebold}

%\setbeamerfont{title}{shape=\itshape,family=\rmfamily}
%\setbeamercolor{title}{fg=red!80!black}
%\setbeamercolor{title}{fg=red!80!black,bg=red!20!black}

% Comment these out if you don't want a slide with just the
% part/section/subsection/subsubsection title:
\AtBeginPart{
  \let\insertpartnumber\relax
  \let\partname\relax
  \frame{\partpage}
}
%\AtBeginSection{
%  \let\insertsectionnumber\relax
%  \let\sectionname\relax
%  \frame{\sectionpage}
%}
%\AtBeginSubsection{
%  \let\insertsubsectionnumber\relax
%  \let\subsectionname\relax
%  \frame{\subsectionpage}
%}

\setlength{\parindent}{0pt}
\setlength{\parskip}{6pt plus 2pt minus 1pt}
\setlength{\emergencystretch}{3em}  % prevent overfull lines
\setcounter{secnumdepth}{0}

\title{Contraction Optimization Summary}
\author{Austin Flynt}
\date{August 8, 2019}

\begin{document}


\begin{frame}
  \titlepage
  %\frametitle{\title}
\end{frame}


\begin{frame}
  \frametitle{Outline}
  \tableofcontents
\end{frame}

\section{Introduction}\label{intro}
\subsection{}
\begin{frame}[t]
  \frametitle{Overview of the Problem}
  \begin{itemize}
    \itemsep1pt\parskip0pt\parsep0pt
    \item There is a need for automated optimization capabilities
    \item such tools have historically been too \alert{costly} and/or not
          generalized enough
    \item need and integrated platform to automatically test geometry in CFD
  \end{itemize}
\end{frame}

\section{Usage}\label{usage}
\subsection{}
\begin{frame}[t]
  \frametitle{Using Optuna}
  \begin{itemize}
    \itemsep1pt\parskip0pt\parsep0pt
    \item 1 what a show that identified in
    \item 2 again that identified in
    \item 3 twice that identified in
  \end{itemize}
\end{frame}

\begin{frame}
  \frametitle{Workflow}
  \begin{columns}[t]
    \begin{column}{0.7\textwidth}
      \adjincludegraphics[width=1.0\linewidth,
      valign=t]{../figures/trialPlot_unsorted.png}
    \end{column}
    \begin{column}{0.5\textwidth}
      \begin{itemize}
        \item define objective function within \texttt{def f(trial):}
      \item specify number of trials, $N_{trials}$
      \item call optimize to execute function $N_{trials}$ times
      \end{itemize}
    \end{column}
  \end{columns}
\end{frame}

% blocks
\begin{frame}[fragile,t]
  \frametitle{Using Star-CCM+}
    \begin{center}
      \includegraphics[height=2.0in]{../figures/trialPlot_unsorted.png}
    \end{center}
    how are there answers
\end{frame}

\begin{frame}
  \frametitle{Two Columns}
  \begin{columns}[c]
    \begin{column}{1.5in}
      \begin{itemize}
        \item Practical \TeX\ 2005\\
        \item Practical \TeX\ 2005\\
        \item Practical \TeX\ 2005
      \end{itemize}
    \end{column}
    \begin{column}{3.0in}
      \framebox{\includegraphics[width=3.0in]{../figures/trialPlot_unsorted.png}}
    \end{column}
  \end{columns}
\end{frame}

\subsection{}
\begin{frame}
  \frametitle{Using Rhino}
  \begin{itemize}
    \itemsep1pt\parskip0pt\parsep0pt
    \item 1 what a show that identified in
    \item 2 again that identified in
    \item 3 twice that identified in
  \end{itemize}
\end{frame}

\subsection{}
\begin{frame}[fragile,t]
  \frametitle{Using Star-CCM+}
  \begin{itemize}
    \itemsep1pt\parskip0pt\parsep0pt
    \item 1 what a show that identified in
    \item 2 again that identified in
    \item 3 twice that identified in
  \end{itemize}
\end{frame}

\section{Testing}\label{testing}
\subsection{}
% blocks
\begin{frame}[fragile,t]
  \frametitle{Using Star-CCM+}
  \begin{block}{Answers}
    how are there answers
  \end{block}
  \pause
  \begin{block}{open questions}
    how are there answers
  \end{block}
\end{frame}

\subsection{}
% nested items
\begin{frame}
  \frametitle{nested items}
  \begin{itemize}
    \item Answered questions
      \begin{itemize}
        \item how may primes?
      \end{itemize}
    \item Open Qs
      \begin{itemize}
        \item is it valid?
      \end{itemize}
  \end{itemize}
\end{frame}


\subsection{}
\begin{frame}
  \frametitle{Whats Still To Do?}
  \begin{columns}
    \column{.5\textwidth}
    \begin{block}{Answered Questions}
      How many primes are there?
    \end{block}
    \pause
    \column{.5\textwidth}
    \begin{block}{Open Questions}
      Is every even number the sum of two primes?
    \end{block}
  \end{columns}
\end{frame}

\subsection{}
\begin{frame}[fragile]
  \frametitle{verbatim code}
  \begin{verbatim}
  int main (void)
  {
    std::vector<bool> is_prime (100,true);
    for( int i = 2; i < 100; i++)
      std::cout << i << "is good" << endl;
  }
  \end{verbatim}
\end{frame}

\subsection{}
\begin{frame}
  \frametitle{Theorem}
  \begin{theorem}
    Average is always less than or equal to maximum value
  \end{theorem}
  \begin{proof}
    \begin{enumerate}
      \item<1-| alert@1> Suppose 1
      \item<2-> Suppose 2
      \item<3-> Suppose 3
      \item<1-> thus x. \qedhere
    \end{enumerate}
  \end{proof}
\end{frame}

\begin{frame}
  \frametitle{Figure}
  \begin{columns}[t]
    \begin{column}{0.7\textwidth}
      \adjincludegraphics[width=1.0\linewidth, valign=t]{../figures/trialPlot_100.png}
    \end{column}
    \begin{column}{0.3\textwidth}
      \textbf{``The problem of distinguishing prime numbers
                from composites, and of resolving composite numbers
                into their prime factors, is one of the most important and
                useful in all of arithmetic."}

      \hfill-- Carl Friedrich Gauss
    \end{column}
  \end{columns}

  \vspace*{10pt}

  \begin{itemize}
    \item Pollard's $p-1$ algorithm (1974)
    \vspace*{10pt}
    \item Quadratic Sieve (QS): Pomerance (1974)
    \vspace*{10pt}
    \item Quadratic Sieve (QS): Pomerance (1974)
  \end{itemize}
\end{frame}


\end{document}
